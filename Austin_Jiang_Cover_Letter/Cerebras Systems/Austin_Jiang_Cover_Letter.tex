\documentclass[11pt]{letter}

\usepackage[letterpaper,margin=1in]{geometry}
\usepackage[hidelinks]{hyperref}
\usepackage[T1]{fontenc}
\usepackage{lmodern}

% Force left alignment
\longindentation=0pt
\setlength{\parindent}{0pt}
\setlength{\parskip}{0.6em}

\signature{Austin Jiang}

\begin{document}

{\Large \textbf{Austin Jiang}}\\
\href{mailto:a68jiang@uwaterloo.ca}{a68jiang@uwaterloo.ca} \textbar\ 
\href{https://github.com/AustinBoyuJiang}{github.com/AustinBoyuJiang} \textbar\ 
\href{https://www.linkedin.com/in/austin-boyu-jiang}{linkedin.com/in/austin-boyu-jiang}

\vspace{0.8em}

\textbf{Dear Hiring Manager,}

I am a Computer Science student at the University of Waterloo applying for the \textbf{AI Software Engineering Intern (Stack, Runtime, MLE)} role at Cerebras Systems. I am particularly drawn to this position because it sits at the intersection of compilers, runtime systems, and performance critical machine learning infrastructure.

I am currently an undergraduate research assistant in the Multicore Lab at Waterloo, where I work on extending Verlib (PPoPP 2024) with \textbf{lock free GPU data structures for range queries}. My work involves designing and validating performance critical components, reasoning about concurrency and memory consistency, and debugging subtle correctness issues by comparing GPU implementations against carefully constructed CPU baselines. This experience closely mirrors the challenges of translating high level abstractions into efficient and correct execution on specialized hardware.

Previously, at Wolfram Research, I worked on parallel cellular automata systems, where I designed tiled execution schemes, halo exchange mechanisms, and synchronization strategies to ensure correctness and scalability. This required profiling system behavior, identifying performance bottlenecks, and iterating on low level execution strategies under real world constraints.

Alongside systems research, I have a strong algorithmic foundation from competitive programming, having placed \textbf{top 10 nationally in the Canadian Computing Olympiad twice} and achieved \textbf{USACO Platinum}. This background supports my ability to reason rigorously about data structures, algorithms, and performance tradeoffs in complex systems.

I am excited about the opportunity to contribute to Cerebras' core software stack and to work closely with compiler, runtime, and machine learning engineers on building reliable and high performance AI systems. I would welcome the chance to apply my systems background and performance focused mindset to this role.

\vspace{0.4em}
Sincerely,\\
Austin Jiang

\end{document}
