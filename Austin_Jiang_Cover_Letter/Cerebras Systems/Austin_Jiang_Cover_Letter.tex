\documentclass[11pt]{letter}

\usepackage[letterpaper,margin=1in]{geometry}
\usepackage[hidelinks]{hyperref}
\usepackage[T1]{fontenc}
\usepackage{lmodern}

% Force left alignment
\longindentation=0pt
\setlength{\parindent}{0pt}
\setlength{\parskip}{0.6em}

\signature{Austin Jiang}

\begin{document}

{\Large \textbf{Austin Jiang}}\\
\href{mailto:a68jiang@uwaterloo.ca}{a68jiang@uwaterloo.ca} \textbar\ 
\href{https://github.com/AustinBoyuJiang}{github.com/AustinBoyuJiang} \textbar\ 
\href{https://www.linkedin.com/in/austin-boyu-jiang}{linkedin.com/in/austin-boyu-jiang}

\vspace{0.8em}

Dear Cerebras Systems recruiting team,

I am a Computer Science student at the University of Waterloo applying for the AI Software Engineering Intern (Stack, Runtime, MLE) role. I am particularly passionate about performance, runtime systems, and machine learning infrastructure, especially work that bridges systems engineering and large-scale ML execution.

I am currently an undergraduate research assistant in the Multicore Lab at Waterloo, where I work on extending Verlib (PPoPP’24) with **lock-free GPU data structures** for efficient range queries. My work focuses on correctness and performance under concurrency, including implementing CPU baselines for controlled benchmarking and debugging low-level GPU systems code.

Previously at Wolfram Research, I worked on a parallel cellular automata runtime, focusing on execution design, performance optimization, and benchmarking. I designed tiled update and halo exchange mechanisms, implemented sparse and dense execution paths, and built benchmarking tools to evaluate scalability and runtime behavior.

In addition, my competitive programming background (two-time Canadian Computing Olympiad top 10 and USACO Platinum) reflects a solid understanding of data structures, algorithms, and systems fundamentals, which I apply consistently in systems and performance engineering work.

Sincerely,\\
Austin Jiang


\end{document}
