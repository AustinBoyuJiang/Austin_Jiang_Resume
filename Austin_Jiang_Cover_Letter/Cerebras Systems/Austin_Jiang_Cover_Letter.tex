\documentclass[11pt]{letter}

\usepackage[letterpaper,margin=1in]{geometry}
\usepackage[hidelinks]{hyperref}
\usepackage[T1]{fontenc}
\usepackage{lmodern}

% Force left alignment
\longindentation=0pt
\setlength{\parindent}{0pt}
\setlength{\parskip}{0.6em}

\signature{Austin Jiang}

\begin{document}

{\Large \textbf{Austin Jiang}}\\
\href{mailto:a68jiang@uwaterloo.ca}{a68jiang@uwaterloo.ca} \textbar\ 
\href{https://github.com/AustinBoyuJiang}{github.com/AustinBoyuJiang} \textbar\ 
\href{https://www.linkedin.com/in/austin-boyu-jiang}{linkedin.com/in/austin-boyu-jiang}

\vspace{0.8em}

Dear Cerebras Systems recruiting team,

I am a Computer Science student at the University of Waterloo applying for the AI Software Engineering Intern (Stack, Runtime, MLE) role. I am deeply passionate about \textbf{performance-critical runtime systems} and \textbf{machine learning infrastructure}, especially software that translates high-level abstractions into efficient and correct execution on specialized hardware.

I am currently an undergraduate research assistant in the Multicore Lab at Waterloo, where I contribute to extending Verlib (PPoPP’24) with \textbf{lock-free data structures on GPU} for efficient range queries. My work emphasizes \textbf{correctness and performance under concurrency}, including reasoning about memory consistency, implementing a \textbf{CPU baseline for controlled benchmarking}, and debugging subtle correctness issues in low-level GPU systems code. I also migrated the project to the \textbf{CUDA toolchain (nvcc)} and designed abstractions supporting both \textbf{versioned and non-versioned} data structures.

In parallel, I conducted research at Wolfram Research on a cellular automata \textbf{runtime and parallel execution engine}. I designed tiled update schemes and halo exchange mechanisms to ensure correctness, implemented sparse frontier updates combined with dense scans achieving a \textbf{127\% speedup} on low-activity states, and built a \textbf{benchmark harness} to measure throughput, active ratios, and scheduling overhead. This work required profiling system behavior and iterating on execution strategies across multiple kernels.

Additionally, my competitive programming background, including placing \textbf{top 10 nationally in the Canadian Computing Olympiad twice} and achieving \textbf{USACO Platinum}, reflects a \textbf{solid understanding of data structures, algorithms, and systems fundamentals}, which I consistently apply to systems and performance engineering problems.

I am excited about the opportunity to contribute to Cerebras’ compiler, runtime, and ML framework stack, and to help build reliable, high-performance software for large-scale AI training systems.

Sincerely,\\
Austin Jiang

\end{document}
